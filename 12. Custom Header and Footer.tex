\documentclass{article}

% fancyhdr package is used for creating header and footer
% for the document
\usepackage{fancyhdr}
% when we use "fancyhdr" package, we have to set our pagestyle
% to fancy using the below command
\pagestyle{fancy}

% by default, there will appear a horizontal line below
% the header. If this command is used, then the horizontal 
% line vanishes.
\renewcommand{\headrulewidth}{0 pt}

\lhead{Left Header}
\chead{Center Header}
\rhead{Right Header}

\lfoot{Left Footer}
\cfoot{Center Footer}
\rfoot{Right Footer}

% for creating a custom header and footer, we have to use 
% \fancypagestyle{style-name-alias}{configuration}
\fancypagestyle{style2}{
\fancyhf{}
% \fancyhead[which-header]{what to write in the header}
\fancyhead[l]{Team Hridoy}
\fancyhead[c]{Quantum Computing}
\fancyhead[r]{IBM}
% \fancyfoot[which-footer]{what to write in the footer}
\fancyfoot[l]{GitHub}
\fancyfoot[c]{Page 3}
\fancyfoot[r]{Twitter}
}

\begin{document}
This is a testing document about headers and footers.
\newpage
% in this second page, the header and footer will remain
% the same just as page 1. Because we didn't modify headers and footers for the next pages.
This is the second page.
\newpage
% if we want a custom header and footer we have to use
% \thispagestyle{} command before the page in which we want the
% customized header and footer and include the style name alias 
\thispagestyle{style2}
This is the page where custom header and footer will appear.
\end{document}
